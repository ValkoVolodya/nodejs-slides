\documentclass{article}
\usepackage{hyperref}
\usepackage[utf8]{inputenc}
\usepackage[english,ukrainian]{babel}
\setcounter{secnumdepth}{0}
\title{\LARGE{Intro to node.js}}
\author{Volodya Valko}
\date{October 2015}
\begin{document}
  \maketitle
  \section{JavaScript}
    \subsection{Using JS on a front-end and back-end}
      Використання JS як на фронт-енді, так і на бек-енді зменшує невідповідності
      між середовищами, а також дозволяє розподілити деяку логіку між клієнтом та
      сервером(наприклад перевірку форм).
    \subsection{JSON}
      Не потрібно кастити об’єкти з бек-енду щоб відправити на фронт і навпаки.
      Все має єдиний формат JSON.
    \subsection{People already know it}
      JavaScript --- це бе перебільшення найбільш популярна мова програмування,
      навіть люди, які не є програмістами можуть знати JS. А якщо навіть котрийсь
      брутальний бекендщик все життя писав на PHP чи Ruby, то node.js стане
      поштовхом для вивченя JS.
  \section{Event-driven development}
    \subsection{Async code}
      Все побудовано на принципах асинхронності, ми нічого не чекаємо, а йдемо напролом
    \subsection{No multithreading, deadlocks}
      Немає жодних проблем зв’язаних з багатопоточністю(але з’являються нові, та
      про це трохи пізніше)
    \subsection{Single thread for your code}
      node.js виконує весь ваш код в одному потоці, а ішу роботу в багато потоків
      (тут має бути наочний приклад, який довго описувати)
  \section{Extremily Fast}
    \subsection{V8 Engine}
      Побудований на швидкому V8 як і Chrome
    \subsection{Optimized for hight concurrent environments}
      Легко масштабувати до тисяч активих з’єднань. Дуже швидкий та ефективний. Може
      в десятки разів зменшити кількість серверних ресурсів в порівнянні з Ruby чи PHP
  \section{npm}
    \subsection{Huge community with tons of great libraries}
      Коли ви бачите ці три маленькі букви --- уявляйте собі лавину --- приблизно
      так розвивається дана спільнота. Все у вільному доступі і лежить на відстані
      однієї команди в терміналі
    \subsection{Five versions of everything}
      Жодних стандартів тут немає. Ваш досвід та враження від написання коду в
      більшості визначаються вибраними вами бібліотеками, фреймами та модулями
  \section{npm must-have modules}
    \subsection{Asynchronous}
      Всі модулі даного розділу покликані звільнити вас від 20 рівнів асинхронних викликів
    \subsection{Web}
      Фреймворки даного розділу спрощують життя і включають в себе template engine
    \subsection{Testing}
      Без фреймворків даного типу в JS далеко не заїдеш. Наша улюблена асинхронність
      та специфіка самого JS робить перевірки дебагом практично повністю неефективними
  \section{Why I write this}
      Це доповідь для презентації під назвою 'Intro to node.js' яку ви можете
      знайти за \href{http://valkovolodya.github.io/nodejs-slides/}{посиланням}.
      Також, ви можете знайти сирцевий код презентації та цього документу за
      \href{https://github.com/ValkoVolodya/nodejs-slides}{посиланням}.
\end{document}
